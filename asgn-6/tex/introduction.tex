\subsection*{Question a}

Since the graph is connected, this means, by definition, that there must be a
path between any given pair of nodes, this also means that for any node, there
must have an edge connected to it. So we get the inequality $n \leq m$. We want
to use this to show $O(n+m)=O(m)$. So let $f \in O(n+m)$:
%
\begin{align*}
  f \in O(n + m) & \Leftrightarrow
  \exists c \in \mathbb{N} \colon \exists N \in \mathbb{N} \colon \forall n \geq N \colon \forall m \geq N \colon
  f\ n\ m \leq c \cdot ( n + m )
\end{align*}
%
So instantiate the quantifiers with $c$, $N$, $n$ and $m$ respectively:
%
\begin{align*}
  f\ n\ m & \leq c \cdot ( n + m ) \leq c \cdot ( m + m ) \leq c \cdot 2 \cdot m
\end{align*}
%
So $2 \cdot c$, $N$, $n$, and $m$ demonstrate that the following holds:
%
\begin{align*}
    \exists c \in \mathbb{N} \colon \exists N \in \mathbb{N} \colon \forall n \geq N \colon \forall m \geq N \colon
    f\ n\ m \leq c \cdot m \Leftrightarrow f \in O(m)
\end{align*}
%
And in the other direction, let $f \in O(m)$:
%
\begin{align*}
  f \in O(m) & \Leftrightarrow
  \exists c \in \mathbb{N} \colon \exists N \in \mathbb{N} \colon \forall n \geq N \colon \forall m \geq N \colon
  f\ n\ m \leq c \cdot m
\end{align*}
%
So instantiate the quantifiers with $c$, $N$, $n$ and $m$ respectively:
%
\begin{align*}
  f\ n\ m & \leq c \cdot m \leq c \cdot ( n + m )
\end{align*}
%
So $c$, $N$, $n$, and $m$ demonstrate that the following holds:
%
\begin{align*}
  \exists c \in \mathbb{N} \colon \exists N \in \mathbb{N} \colon \forall n \geq N \colon \forall m \geq N \colon
  f\ n\ m \leq c \cdot (n + m) \Leftrightarrow f \in O(n + m)
\end{align*}
%
And so we conclude that $O(n+m) = O(m)$ as desired. \QED
%
\subsection*{Question b}
%
For this question I'll reason a little bit less formally - leaving out the
quantifiers needed to establish membership of big-oh.

The inclusion $O(m\lg n) \subseteq O(m\lg m)$ follows trivially from $n \leq m$. In the other case we have:
%
\begin{align*}
  m \cdot \lg m \leq m \cdot \lg n^2 \leq m \cdot 2 \cdot \lg n
\end{align*}
%
And so the other inclusion holds as well leaving us to conclude $O(m\lg n)
= O(m\lg m)$. \QED
%
\subsection*{Question c}
This statement is not true. The following graph is a counter-example:
$G=(V,E)$:
%
\begin{align*}
  V & = \{0,1,2,3\} \\
  E & = \{\{0,1\},\{0,2\},\{0,3\},\{1,2\},\{2,3\}\}
\end{align*}
%
Here's a sample illustration of the graph:
%
\begin{verbatim}
0 - 1
| \ |
3 - 2
\end{verbatim}
%
Here, $\{0,1,2,3\}$ is a minimum spanning tree (i.e. the result of a possible
depth-first search), but $\{0,2,3\}$ is a path with two edges not in this tree.

I'm pretty sure I don't understand the second part of this question. An
algorithm for finding cycles in a graph could go like this: Pick a node, do
depth-first search, saving each node we visit along the way in a list. If we
encounter a node that we have seen before, remove all elements up to this node.
Return the list.
%
\subsection*{Question d}
My algorithm goes like this: Pick a node, perform depth-first search on this
while coloring nodes using the following system: A color is chosen for the first
node. Each subsequent new node we visit gets colored the opposite of the node
preceding it. If we try to visit a node we have already colored, if that color
is inconsistent with what we would've colored it, report that the graph is not
bipartite.

This algorithm is correct because each time we visit a new node the bipartite
property requires that the new node must have a different color. And since the
problem is symmetric (a 2-coloring is valid if and only if the ``inverted''
two-coloring is valid) we can not make a ``wrong'' chioce at any point.
